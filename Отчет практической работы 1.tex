\documentclass[12pt,a4paper]{article}
\usepackage[utf8]{inputenc}
\usepackage[russian]{babel}
\usepackage[left=2.00cm, right=2.00cm, top=2.00cm, bottom=2.00cm]{geometry}
\linespread{1.25}
\usepackage{setspace}
\usepackage{indentfirst}
\setlength{\parindent}{1.25cm}
\let\paragraph\ignorespaces
\usepackage{tabularx}
\usepackage{multirow}
\usepackage{graphicx}



\begin{document}
	
\begin{titlepage}
	
\begin{center}
	\large Университет ИТМО\\[5cm]
	\LARGE Практическая работа №1\\
	\normalsize по дисциплине <<Визуализация и моделирование>>\\[5cm]
\end{center}
\begin{flushright}
		\begin{minipage}{0.6\textwidth}
		\begin{flushleft}
			\large
			\singlespacing 
			\textbf{Автор:} Хоанг Минь Тханг\\
			\textbf{Поток:} ВИМ1\\
			\textbf{Группа:} К3220\\
			\textbf{Факультет:} ИКТ\\
			\textbf{Преподаватель:} Чернышева А.В.
		\end{flushleft}
	\end{minipage}
\end{flushright}

\vfill

\begin{center}
	{\large Санкт-Петербург, \the\year{ г.}}
\end{center}
 
\end{titlepage}
\normalsize

Описание: Датасет Airline Safety содержит информацию об авариях от каждой авиакомпании в США.

\begin{center}
    \resizebox{\columnwidth}{!}{\begin{tabular}{||l|l|l||}
    \hline
    Название столбца в датасете & Определение & Тип данных\\
    \hline
    airline & Название авиакомпании & Текст\\
    avail\_seat\_km\_per\_week & Кресло-километр каждую неделю & Целое число\\
    incidents\_85\_99 & Итоговое количество инцидентов 1985-1999 & Целое число\\
    fatal\_accidents\_85\_99 & Итоговое количесто инцидентов со смертельным исходом 1985-1999 & Целое число\\
    fatalities\_85\_99 & Итоговое количество погибщих 1985-1999 & Целое число\\
    incidents\_00\_14 & Итоговое количество инцидентов 2000-2014 & Целое число\\
    fatal\_accidents\_00\_14 & Итоговое количесто инцидентов со смертельным исходом 2000-2014 & Целое число\\
    fatalities\_00\_14 & Итоговое количество погибщих 2000-2014 & Целое число\\
    \hline
    \end{tabular}}
    
Задача решать в этом датасете: определение безопасности каждой авиакомпании 
\end{center}













\end{document}